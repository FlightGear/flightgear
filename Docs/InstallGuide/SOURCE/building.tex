%%
%% getstart.tex -- Flight Gear documentation: Installation and Getting Started
%% Chapter file
%%
%% Written by Michael Basler % Bernhard Buckel, starting September 1998.
%%
%% Copyright (C) 1999 Michael Basler (pmb@knUUt.de)
%%                  & Bernhard Buckel (buckel@wmad95.mathematik.uni-wuerzburg.de)
%%
%% This program is free software; you can redistribute it and/or
%% modify it under the terms of the GNU General Public License as
%% published by the Free Software Foundation; either version 2 of the
%% License, or (at your option) any later version.
%%
%% This program is distributed in the hope that it will be useful, but
%% WITHOUT ANY WARRANTY; without even the implied warranty of
%% MERCHANTABILITY or FITNESS FOR A PARTICULAR PURPOSE.  See the GNU
%% General Public License for more details.
%%
%% You should have received a copy of the GNU General Public License
%% along with this program; if not, write to the Free Software
%% Foundation, Inc., 675 Mass Ave, Cambridge, MA 02139, USA.
%%
%% $Id: getstart.tex,v 0.20 1999/06/04 michael
%% (Log is kept at end of this file)

%%%%%%%%%%%%%%%%%%%%%%%%%%%%%%%%%%%%%%%%%%%%%%%%%%%%%%%%%%%%%%%%%%%%%%%%%%%%%%%%%%%%%%%%%%%%%%%%%
\chapter{Building the plane: Compiling\index{compiling} the program\label{building}}
%%%%%%%%%%%%%%%%%%%%%%%%%%%%%%%%%%%%%%%%%%%%%%%%%%%%%%%%%%%%%%%%%%%%%%%%%%%%%%%%%%%%%%%%%%%%%%%%%
\markboth{\thechapter.\hspace*{1mm} BUILDING THE
PLANE}{\thesection\hspace*{1mm} COMPILING UNDER LINUX}

This central Chapter describes how to build \FlightGear on several systems. In case you
are on a Win32 (i.\,e. Windows 98 or Windows NT) platform you may not want to go though
that potentially troublesome process but instead skip that Chapter and straightly go to
the next one. (Not everyone wants to build his or her plane himself or herself, right?)
However, there may be good reason at least to try building the simulator:

\begin{itemize}
\item In case you are on a \Index{UNIX}/\Index{Linux} platform there are supposedly no
pre-compiled binaries\index{binaries, pre-compiled} available for your system. We do not
see any reason why the distribution of pre-compiled binaries (with statically linked
libraries) should not be possible for \Index{UNIX} systems in principle as well, but in
practice it is common to install programs like this one on \Index{UNIX} systems by
recompiling them.

\item There are several options you can set only during
compile time. One such option is the decision to compile with
hardware or software \Index{OpenGL} rendering enabled. A more
complete list goes beyond this \textit{Installation and Getting
Started} and should be included in a future
\textit{\Index{\FlightGear Programmer's Guide}}.

\item You may be proud you did.
\end{itemize}

On the other hand, compiling \FlightGear is not a task for novice users. Thus, if you're
a beginner (we all were once) we recommend postponing this and just starting with the
binary distribution to get you flying.

Besides, there have been two branches of code starting from version 0.6. For more
details, see Section \ref{branches}. This description generally refers to the stable,
even-numbered branch. It is almost certain, that odd-numbered versions require
modifications to that.

As you will note, this Chapter is far from being complete. Basically, we describe
compiling for two operating systems only, \Index{Windows 98/NT} and \Index{Linux}. There
is a simple explanation for this: These are just the systems we are working on. We hope
to be able to provide descriptions for more systems based on contributions written by
others.

\section{Compiling\index{compiling!Linux} under \Index{Linux}}

If you are running Linux you probably have to build your own
\Index{binaries}. The following is one way to do so.

\begin{enumerate}

%%Bernhard 25.06.1999

\item \FlightGear needs some supplementary libraries which are usually
  not contained in any distribution we know of. These are:

  \begin{itemize}

  \item{{\em plib}} which is absolutely essential for the building
    process. Get the latest version of {\em plib} at
    \web{http://www.woodsoup.org/projs/plib/} and follow the
    instructions contained in README.plib.

  \item{{\em gfc}} is only needed if you want to build the scenery
    generation tools but it doesn't hurt to have it installed. It can
    be found along with the building instructions at
    \web{http://www.geog.psu.edu/~qian/gfc/index.html}.

  \item{{\em gpc}} which is also needed for the scenery generation
    tools. Get it from
    \web{http://www.cs.man.ac.uk/aig/staff/alan/software/}, building
    instructions can be found in README.gpc in your \FlightGear source
    directory.

  \end{itemize}

  Now you are ready to proceed to the task of getting, compiling and installing \FlightGear itself:

\item Get the file \texttt{FlightGear-x.xx.tar.gz} from the
  \texttt{source} subdirectory under

 \web{ftp://ftp.flightgear.org/pub/fgfs/Source/}

 \noindent

\item Unpack it using :

        \texttt{tar xvfz FlightGear-x.xx.tar.gz}.

\item \texttt{cd} into \texttt{FlightGear-x.xx}. Run:

        \texttt{./configure}

 \noindent
and wait a few minutes. \Index{configure} knows about a lot of
options. Have a look at the file \texttt{INSTALL} in the
\FlightGear source directory to learn about them. If run without
options, configure assumes that you will install the data files
under \texttt{/usr/local/lib/FlightGear}.


\item Assuming configure finished successfully, simply run

        \texttt{make}

 \noindent
and wait for the make process to finish.


\item Now become root (for example by using the su command) and
type

        \texttt{make install}.

 \noindent
This will install the \Index{binaries} in \texttt{/usr/local/bin}.

There is a problem concerning permissions under Linux/Glide. All programs accessing the
accelerator board need root permissions. The solution is either to play as root (this is
{\em bad} practice and not always possible) or make the \texttt{/usr/local/bin/fgfs}
binary \texttt{setuid root}, i.e. when this binary is run root privileges are given. Do
this by issuing (as root)

   \texttt{chmod +s /usr/local/bin/fgfs}.

 \noindent

 Again, this is a quick and dirty hack.  The perfect solution for this
 problem is using a kernel module called {\em 3dfx.o}. It is available
 along with documentation at \web{http://www.xs4all.nl/~carlo17/3dfx/index.html}
 and it might be a good idea to read some of the Quake-related links there!

 To install this kernel module, just download it, become root and
 issue the following commands:

\texttt{mkdir dev3dfx}

\texttt{cd dev3dfx}

\texttt{tar xvfz ../Dev3Dfx-2.7.tar.gz}

\texttt{make}

\texttt{cp 3dfx.o /lib/modules/`uname -r`/misc}

\texttt{mknod /dev/3dfx c 107 0}

\texttt{insmod 3dfx}

It is a good idea to put the last line into one of your bootup scripts! After having
installed this module, you can even go ahead and remove the S-bit from {\em all} programs
that need access to your 3D hardware.


\end{enumerate}

\section{Compiling\index{compiling!Windows 98/NT} under \Index{Windows 98/NT}}

\begin{enumerate}
\item Windows, contrary to Linux which brings its own compiler, comes
not equipped with developmental tools. Several compilers have been shown to work for
compiling {\FlightGear}, including the \Index{Cygnus Win32 port of GNU C}++ and the
\Index{MS Visual C} compiler. Given that the project will be a free one we prefer the
Cygnus Compiler as it provides a free development environment. However, we will be happy
to include a proper description in case those who worked out how to compile with MSVC or
other Compilers provide one.

\item  Install and configure the \Index{Cygnus} Gnu-Win32 development
    environment. The latest version is Beta 20. The main
    Cygnus Gnu-Win32 page is at:

        \web{http://sourceware.cygnus.com/cygwin/}.

 \noindent
    You can download the complete Cygnus Gnu-Win32 compiler from:

        \web{ftp://go.cygnus.com/pub/sourceware.cygnus.com/cygwin/latest/full.exe}.

    Be sure to read this package's README files to be found under the main page, first.

 \noindent
    To install the compiler, just run \texttt{full.exe} by double-clicking in
    Windows explorer.  After doing so you'll find a program group called
    \texttt{Cygnus Solutions} in your Start menu. Do not forget making a copy of the
    shell under c:/bin, as detailed in the docs.

\item  Open the Cygnus shell via its entry in the Start menu.
    Mount the drive where you want to build \FlightGear as follows
    (assuming your \FlightGear drive is \texttt{d:}):

         \texttt{mkdir /mnt}\\
         \texttt{mount d: /mnt}

 \noindent
    You only have to do this once. The drive stays mounted (until you
    umount it) even through reboots and switching off the machine.


\item Before actually being able to compile \FlightGear you have to install a hand full
of support libraries required for building the simulator itself. Those go usually into
\texttt{c:/usr/local} and it is highly recommended to choose just that place.

 First, you have to install the free \Index{win32 api library} (the latest
 version being 0.1.5). Get the package \texttt{win32api-0.1.5.tar.gz} from:

    \web{http://www.acc.umu.se/~anorland/gnu-win32/w32api.html}

Conveniently you may unpack the package just onto you \FlightGear drive. Copy the file to
the named drive, open the Cygnus shell via the Start menu entry and change to the
previously mounted drive with

 \texttt{cd /mnt}

 Now, you can unpack the distribution with

 \texttt{gzip -d win32api-0.1.5.tar.gz}\\
 \texttt{tar xvf win32api-0.1.5.tar}

 This provides you with a directory containing the named libraries. For installing them,
 change to that directory with

\texttt{cd win32api-0.1.5}

and type

 \texttt{make}\\
 \texttt{make install}

This installs the libraries to their default locations under \texttt{c:/usr/local}

\item To proceed, you need the \Index{glut libraries}. Get these from the same site named
above

\web{http://www.acc.umu.se/~anorland/gnu-win32/w32api.html}

as \texttt{glutlibs-3.7beta.tar.gz}. Just copy the package to your \FlightGear drive and
unpack it in the same way as describes above. There is no need to run \texttt{make} here.
Instead, just copy the two libraries \texttt{libglut.a} and \texttt{libglut32.a} to
\texttt{c:/usr/local/lib}. There is no need for the two accompanying \texttt{*.def} files
here.

\item Next, get the \Index{Glut header files}, for instance, from

\web{ftp:://ftp.flightgear.org/pub/fgfs/Win32/Mesa-3.0-includes.zip}

Unpack these as usual with \texttt{unzip -d} and copy the contents of the resulting
directory \texttt{/gl} to \texttt{c:/usr/local/include/gl}

\item Finally, you need Steve Backer's \Index{PLIB} being one of the key libraries for \FlightGear\hspace{-1mm}.
Get the most recent version \texttt{plib-X.X.tar.gz} from

\web{http://www.woodsoup.org/projs/plib/}

(There are mirrors, but make sure they contain the most recent version!). Copy it to your
\FlightGear drive, open the Cygnus shell and unpack the library as described above.

Next, change into \Index{PLIB}'s directory. It is recommended to configure \Index{PLIB}
with the following command line (you can make a script as I did if it hurts)

\begin{ttfamily}
CFLAGS="-O2 -Wall" CXXFLAGS="-O2 -Wall"\\ CPPFLAGS=-I/usr/local/include
LDFLAGS=-L/usr/local/lib ./configure
--prefix=/usr/local\\
 --includedir=/usr/local/include/plib
\end{ttfamily}

You must write all this \textbf{on one line} without any line breaks in between!

Finally, build \Index{PLIB} with

 \texttt{make}\\
 \texttt{make install}

\item Now, you're finally prepared to build \FlightGear itself.

 Fetch the \FlightGear code and special \Index{Win32 libraries}.  These
can be found at:


 \web{ftp://ftp.flightgear.org/pub/fgfs/Source/}

 \noindent
    Grab the latest \texttt{FlightGear-X.XX.zip} and
    \texttt{win32-libs-X.XX.zip} files.

(It you're really into adventures, you can try one of the recent snapshots instead.)

\item Unpack the \FlightGear source code via

        \texttt{pkunzip -d FlightGear-X.XX.zip}.

 \noindent

\item  Change to the newly created \texttt{FlightGear-X.XX directory} with e.\,g.

\texttt{cd //D/FlightGear-X.XX}

 and unpack the Win32 libraries there:

     \texttt{pkunzip -d win32-libs-X.XX.zip}.


\item  You will find a file called \texttt{install.exe} in the Win32
directory after unzipping \texttt{win32-libs-X.XX.zip}. This
version of \texttt{install.exe} should replace the one in your
$\backslash$\texttt{H-i386-cygwin32$\backslash$bin} directory --
it's sole claim to fame is that it understands that when many
calls to it say \texttt{install foo} they mean \texttt{install
foo.exe}. If you skip this step and attempt an install with the
older version present \texttt{make install} will fail.

Side Note: We need to make a distinction between the
\texttt{\Index{build tree}} and the \texttt{\Index{install tree}}.
The \texttt{build tree} is what we've been talking about up until
this point.  This is where the source code lives and all the
compiling takes place.  Once the executables are built, they need
to be installed someplace.  We shall call this install location
the \texttt{install tree}.  This is where the executables, the
scenery, the textures, and any other run-time files will be
located.

\item \Index{Configure} the make system for your environment and your
\texttt{install tree}. Tell the configure script where you would like to install the
\Index{binaries} and all the \Index{scenery} and \Index{textures} by using the
\texttt{-$\!$-prefix} option. In the following example the base of the \texttt{install
tree} is \texttt{FlightGear}. Make sure you are within \FlightGear's \texttt{build tree}
root directory.

\item Run:\index{configure}

        \texttt{./configure -$\!$-prefix=/mnt/FlightGear}.

 \noindent
Side note: The make procedure is designed to link against opengl32.dll, glu32.dll, and
glut32.dll which most accelerated boards require. If this does not apply to yours or if
you installed SGI's \Index{software rendering} as mentioned in Subsection \ref{softrend}
you may have to change these to opengl.dll, glu.dll, and glut.dll. (In case you're in
doubt check your \texttt{$\backslash$windows$\backslash$system} directory what you've
got.)

 If this is the case for your \Index{video card}, you can edit
 \texttt{.../Simulator/Main/ Makefile} and rename these three libraries to
    their "non-32" counterparts.  There is only one place in this
    \texttt{Makefile} where these files are listed.

\item Build the executable.  Run:

        \texttt{make}.

Assuming you have installed the updated version of \texttt{install.exe} (see earlier
instructions) you can now create and populate the \texttt{install tree}.  Run:

        \texttt{make install}.

    You can save a significant amount of space by stripping all the
    debugging symbols off of the executable.  To do this, change to the
    directory in the \texttt{install tree} where your binary lives and run:

    \texttt{strip fgfs.exe} resp. \texttt{strip fgfs-sgi.exe}.
  \end{enumerate}

%% Revision 0.00 1998/09/08 michael
%% Initial revision for version 0.53.
%% employing redame.win32/readame.linux
%% by c. olson , b. buckel
%% Revision 0.01 1998/09/20 michael
%% several extensions and corrections
%% revision 0.10 1998/10/01 michael
%% final proofreading for release
%% revision 0.11 1998/11/01 michael
%% deleted some obsolete stuff from the Linux Section
%% revision 0.12 1999/03/07 michael
%% changed Windows to Cygnus b20
%% revision 0.20 1999/06/04 michael
%% complete rewrite of the windows build Section exploiting Curt's README.win32
%% revision 0.21 1999/06/30 bernhard
%% complete rewrite of Linux build Section
