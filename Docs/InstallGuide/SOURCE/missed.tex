%%
%% getstart.tex -- Flight Gear documentation: Installation and Getting Started
%% Chapter file
%%
%% Written by Michael Basler % Bernhard Buckel, starting September 1998.
%%
%% Copyright (C) 1999 Michael Basler (pmb@knUUt.de)
%%                  & Bernhard Buckel (buckel@wmad95.mathematik.uni-wuerzburg.de)
%%
%% This program is free software; you can redistribute it and/or
%% modify it under the terms of the GNU General Public License as
%% published by the Free Software Foundation; either version 2 of the
%% License, or (at your option) any later version.
%%
%% This program is distributed in the hope that it will be useful, but
%% WITHOUT ANY WARRANTY; without even the implied warranty of
%% MERCHANTABILITY or FITNESS FOR A PARTICULAR PURPOSE.  See the GNU
%% General Public License for more details.
%%
%% You should have received a copy of the GNU General Public License
%% along with this program; if not, write to the Free Software
%% Foundation, Inc., 675 Mass Ave, Cambridge, MA 02139, USA.
%%
%% $Id: getstart.tex,v 0.20 1999/06/04 michael
%% (Log is kept at end of this file)

%%%%%%%%%%%%%%%%%%%%%%%%%%%%%%%%%%%%%%%%%%%%%%%%%%%%%%%%%%%%%%%%%%%%%%%%%%%%%%%%%%%%%%%%%%%%%%%%%
\chapter{Missed approach: If anything refuses to work\label{missed}}
%%%%%%%%%%%%%%%%%%%%%%%%%%%%%%%%%%%%%%%%%%%%%%%%%%%%%%%%%%%%%%%%%%%%%%%%%%%%%%%%%%%%%%%%%%%%%%%%%
\markboth{\thechapter.\hspace*{1mm} MISSED APPROACH
}{\thesection\hspace*{1mm} ???}

We tried to sort \Index{problems} according to operating system to a certain extent , but
if you encounter a problem it may be a wise idea to look beyond ''your'' operating system
-- just in case. Besides, if anything fails, it is definitely a good idea to check
the FAQ maintained by Oliver Delise (\mail{delise@rp-plus.de}) being distributed
along with the source code.

\section{General problems}
\begin{itemize}
\item{\FlightGear runs SOOO slow}\\
 If the \Index{HUD} indicates you are getting something like 1\,fps
 (frame per second) or below you typically don't have working hardware
 \Index{OpenGL} support. There may be several reasons for this. First,
 there may be no OpenGL hardware drivers available for older
 cards. In this case it is highly recommended to get a new board.

 Second, check if your drivers are properly installed. Several
 cards need additional OpenGL support drivers besides the
 ''native'' windows ones. For more detail check Chapter
 \ref{opengl}.

 Third, check if your hardware driver is called \texttt{opengl32.dll}
 or just merely \texttt{opengl.dll}. By the default compilation, binaries are linked against
 \texttt{open} \texttt{gl32.dll}. If you require the non-32 version,
 consider rebuilding \FlightGear with the libraries \texttt{opengl32.dll},
 \texttt{glut32.dll}, and \texttt{glu32.dll} replaced by their
 non-32 counterparts. For more details check Chapter
 \ref{building}.

 If you installed the pre-compiled binaries \texttt{runfgfs.bat} invokes
 \texttt{fgfs.exe} while \texttt{runfgfs.sgi.bat} invokes
 \texttt{fgfs.sgi.exe} with the first ones being linked against the 32-versions.

 Usually, hardware accelerated drivers use the 32-libraries.

 \end{itemize}

\section{Potential problems under Linux}

Since we don't have access to all possible flavors of Linux distributions, here are some
thoughts on possible causes of problems. (This Section includes contributions by Kai
Troester \mail{Kai.Troester@rz.tu-ilmenau.de}.)

\begin{itemize}
  

\item{Wrong library versions}\\
  This is a rather common cause of grief especially when you prefer to
  install the libraries needed by \FlightGear by hand. Be sure that
  especially the Mesa library contains support for the \Index{3DFX
    board} and that \Index{Glide} libraries are installed and can be
  found. If a \texttt{ldd `which fgfs`} complains about missing
  libraries you are in trouble.
  
  You should also be sure to keep \em{always} the \em{latest} version
  of Steve's plib on your system. Lots of people (including me) have
  failed miserably to compile \FlightGear just because of an outdated
  plib.
  

\item{Missing \Index{permissions}}\\
  \FlightGear needs to be setuid root in order to be capable of
  accessing the accelerator board (or a special kernel module as
  described earlier in this document). So you can either issue a

  \texttt{chown root.root /usr/local/bin/fgfs ;}\\
  \texttt{chmod 4755 /usr/local/bin/fgfs}
  
  to give the \FlightGear binary the proper rights or install the 3DFX module. The latter is the ``clean''
  solution and strongly recommended!


\item{Non-default install options}\\
  \FlightGear will display a lot of diagnostics when being started up.
  If it complains about bad looking or missing files, check that you
  installed them in the way they are supposed to be, i.e. latest
  version and proper location. The canonical location \FlightGear
  wants its data files under \texttt{/usr/local/lib}. Be sure to
  grab the latest versions of everything that might be needed!

\item{Compile problems}\\
  Check as far as you can, as a last resort (and a great information
  source, too) there are mailing lists for which information can be
  gotten at

  \web{http://www.flightgear.org/mail.html}.

This will give you direct contact to the developers.

\item{Configure could not find Mesa and Glut though they are
installed}

If the configure script could not find your Mesa and Glut libraries you should add the
Mesa library-path (i.e. \texttt{/usr/local/Mesa}) to the EXTRA\_DIRS variable in the file
configure.in (i.e. \texttt{EXTRA\_DIRS=''/usr/local/usr/}
\texttt{X11R6/usr/local/Mesa''}). After this you have to run autoconf. (Please read
README.autoconf for running autoconf )

\item{SuSE Distribution}
\begin{itemize}
 \item If you have a SuSE distribution use the egcs compiler instead
of the compiler delivered with SuSE. Grab it at

\web{http://egcs.cygnus.com}

 \item SuSE 6.0 users should also use the Glide,
Mesa and Glut Libraries delivered with the distribution
 \item A known problem of Flight Gear until version Version 0.57 with SuSE concerns
  \texttt{acconfig.h}. If 'make' stops and reports an error in relation with acconfig.h
insert the following lines to \texttt{/usr/share/autoconf/} \texttt{acconfig.h}:

        \texttt{/* needed to compile fgfs properly*/}\\
        \texttt{{\#}undef FG\_NDEBUG}\\
        \texttt{{\#}undef PACKAGE}\\
        \texttt{{\#}undef VERSION}\\
        \texttt{{\#}undef WIN32a}

(a solution for this problem is coming soon )
\end{itemize}

  %%B.B. 21.2.99
  Additionally there are two versions of the GNU C compiler around:
  egcs and gcc (the classic one). gcc seems to have its own notion of
  some C++ constructs, so updating to egcs won't hurt and maybe help
  to compile the program.
  %%

\end{itemize}

\section{Potential problems under Windows 98/NT}

\begin{itemize}
\item{The executable refuses to run.}\\
 You may have tried to start the executable directly either by
 double-clicking \texttt{fgfs.exe} in Windows explorer or by invoking it
 in a MS-DOS shell. Double-clicking via explorer does never work
 (except you set the environment variable \texttt{FG\_ROOT}
 in the autoexec.bat or otherwise). Rather double-click \texttt{runfgfs.bat} or
 \texttt{runfgfs-sgi.bat}  For more detail, check Chapter \ref{takeoff}.

 Another potential problem might be you did not download the
 most recent versions of scenery and textures required by \FlightGear, or
 you did not load any scenery or texture at all. Have a close look
 at this, as the scenery/texture format is still under development and may
 change frequently. For more detail, check Chapter \ref{prefligh}.

 A further potential source of trouble are so-called
 \Index{mini-OpenGL} drivers provided by some manufacturers. In this case,
 {\FlightGear}'s typically hangs while opening the graphics window.
 In this case, either replace the \Index{mini-OpenGL} driver by a
 full OpenGL driver or or in case such is not available install
 software OpenGL support (see Section \ref{softrend}).

\item{\FlightGear ignores the command line parameters.}\\
 There is a problem with passing command line options containing a
 ''='' to windows batch files. Instead, include the options into
 \texttt{runfgfs.bat}.

\item{While compiling with the Cygnus Compiler \texttt{Configure}
complains not to find \texttt{glu32.dll}}.

Make sure you change to the Main FlightGear directory, e.\,g. with

\texttt{cd //D/FlightGear-X.XX}

before running \texttt{Configure} and \texttt{Make}. Do not forget the win32 library
package.

\item{I am unable to build \FlightGear under \Index{MSVC}/\Index{MS DevStudio}}\\
 By default, \FlightGear is build with GNU C++, i.\,e. the
 \Index{Cygnus} compiler for Win32. For hints or Makefiles
 required for MSVC for MSC DevStudio have a look into

 \web{http://www.flightgear.org/Downloads/Source}.

In principle, \FlightGear should be buildable with the project files provided.

\item{Compilation of \FlightGear dies not finding \texttt{gfc}}.

The library \texttt{gfc} cannot be build with the Cygnus compiler at present. It us
supposed to be substituted by something else in the future.

As the simulator is already built at this point, you simply can forget about that problem
as long as you don't intend to build the \Index{scenery creation tools}. Just go on with
\texttt{make install}.

\end{itemize}


%% revision 0.10  1998/10/01  bernhard
%% added win stuff michael
%% final proofreading for release
%% revision 0.11  1998/11/01  michael
%% Remark on mini-OpenGL drivers, new general Section
%% Access violation error under win32 added
%% Command line problem in win32 added
%% revision 0.12  1999/03/07  bernhard
%% Remark on EGCS compiler
%% revision 0.12  1999/03/07  michael
%% Added Contribution by Kai Troester
%% Reworked Win32 Stuff
%% revision 0.20  1999/06/04  michael
%% added hint to FAQ, gfc problem
